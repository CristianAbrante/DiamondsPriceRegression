% Geometry: es para modificar los márgenes del documento
\usepackage[left=3in, right=2.5cm, top=2.5cm, bottom=2.5cm]{geometry}
%% lipsum: para generar texto de ejemplo. Se puede eliminar una vez que se eliminen todos los \lipsum[] del documento.
\usepackage{lipsum}
% lscape: en caso de querer rotar una hoja, buscar información en internet en caso de ser requerido. 
\usepackage{lscape}
\usepackage{float}
% inputenc: para que latex acepte caracteres latinos como los acentos y la letra ñ.
\usepackage[utf8]{inputenc}
% babel: para traducir los títulos que vienen originalmente en inglés. Ejemplo: Fecha, Chapter, Bibliography, Appendix, etc.
\usepackage[spanish,es-tabla]{babel}
% natbib: para poder citar utilizando paréntesis redondos con \citep{•} o sin parentesis con \cite{•}
\usepackage[numbers]{natbib}
\usepackage{float} % para usar [H] y obligar que las figuras o tablas aparezcan donde es requerido.
\usepackage[pdftex]{graphicx} % graphicx: para incorporar imágenes. Recordar que las imágenes 'gif' no son aceptadas por Latex, se sugiere utilizar formato png por su calidad, en segunda intantcia jpg.
\usepackage{parskip} % parskip: par no dejar sangrías e insertar espacios entre párrafos en su lugar.
\usepackage{amsmath} %paquete para escribir fórmulas matemáticas.
\usepackage{amsfonts}%paquete para escribir fórmulas matemáticas.
\usepackage{amssymb} %paquete para escribir fórmulas matemáticas.
\usepackage{amsbsy}
\usepackage{upgreek}
\usepackage[usenames,dvipsnames,svgnames,table]{xcolor} %xcolor: para definir colores y dar color a tablas.
\usepackage{multirow}

\graphicspath{ {figures/} }
\usepackage{array}

% hyperref: define opciones especiales para el documento PDF producido.
\usepackage[pdftex, bookmarksnumbered,  pagebackref, colorlinks=true, citecolor=DarkBlue, linkcolor=DarkBlue!30!Black, urlcolor=Black,bookmarksopen]{hyperref}

% El paquete fancyhdr es para definir opciones de encabezado y pié de página
% Según el formato existente a la fecha (agosto de 2015) esto no se considera
% Su utilización en este caso es para situar el número de página en la parte
% inferior derecha de la página.

\usepackage{fancyhdr} % activamos el paquete
	\pagestyle{fancy} % seleccionamos un estilo
	\lhead{} % texto izquierda de la cabecera
	\chead{} % texto centro de la cabecera
	\rhead{\textcolor[gray]{0.5}{\textit{\nouppercase \leftmark}}} % Nombre del capítulo. \nouppercase: uso de minúsculas
	\lfoot{} % texto izquierda del pie
	\cfoot{} % imagen centro del pie
	\rfoot{\textcolor[gray]{0.5}{\thepage}} % Número de página a la derecha, abajo
	\renewcommand{\headrulewidth}{0.2pt} % grosor de la línea de la cabecera

\fancypagestyle{detailed}{
    \fancyhf{} % clear all header and footers
    \fancyfoot[R]{\textcolor[gray]{0.5}{\thepage}}
	%\fancyhead{}    
    \renewcommand{\headrulewidth}{0pt}
 }
 
\usepackage{etoolbox}
\patchcmd{\chapter}{\thispagestyle{plain}}{\thispagestyle{detailed}}{}{}

%times: para uar letra tipo Times New Roman
\usepackage{times}

%separación entre líneas (1.2 espacios). En word es interlineado exacto a 12 pts
\renewcommand{\baselinestretch}{1.2} 

\usepackage{titlesec} % para poder modificar los títulos

% Para la numeración de tablas y figuras.
\renewcommand\thefigure{\arabic{section}.\arabic{figure}} % Genera numeración X.Y
\renewcommand\thetable{\arabic{section}.\arabic{table}} % Genera numeración X.Y
\numberwithin{figure}{section} %Hace que la primera figura de cada sección X sea X.1
\numberwithin{table}{section} %Hace que la primera tabla de cada sección X sea X

\usepackage{booktabs} % Para trabajar con opciones especiales de tablas.
\usepackage{caption}

\usepackage{setspace}

% El índice de tablas e imágenes se superpone el texto al número de la figura o tabla.
% Esta configuración arregla dicho problema, modificar el 3.0 de ser necesario.
\usepackage{tocloft}
\addtolength{\cftfignumwidth}{3.0em}
\renewcommand{\cftfigpresnum}{\figurename\ }
\addtolength{\cfttabnumwidth}{3.0em}
\renewcommand{\cfttabpresnum}{\tablename\ }