\documentclass[a4paper, 9pt]{article}
\usepackage{geometry}
%% \input{configuracion}
\usepackage[utf8]{inputenc}
\usepackage{amssymb}
\usepackage{graphicx}
\usepackage{lscape}
\usepackage{float}
\usepackage{changepage}
\usepackage{capt-of}
\usepackage{wrapfig}

\begin{document}

\input{portada.tex}

\newgeometry{textwidth=18cm,textheight=27cm}

\section{Descripción del conjunto de datos}
\label{sec:descripcion-datos}
\noindent

El conjunto de datos que estudiaremos se trata de una lista de 308 piedras preciosas. Este conjunto de datos se ha creado con fines educativos e incluye las siguientes variables:

\begin{itemize}
    \item \texttt{caratage}: Esta variable numérica continua se refiere a los quilates que posee el diamante, unidad de medida utilizada para pesar gemas y otras piedras preciosas. La correspondencia es de 1 quilate con 0.2 gramos.
    \item \texttt{purity}: Nivel de pureza del color del diamante. Varía desde el nivel \textbf{D} hasta el \textbf{I}, donde el \textbf{D} se corresponde con el nivel más puro.
    \item \texttt{clarity}: Nivel de claridad del diamante. Se corresponde con los valores: \textbf{IF, VVS1, VVS2, VS1, VS2}, ordenados de mejor a peor.
    \item \texttt{certificate}: Certificado de calidad emitido por una institución de renombre. Puede tener los siguientes valores: \textbf{GIA, IGI y HRD}.
    \item \texttt{price}: Precio que tiene el diamante, expresado en dólares de Singapur.
\end{itemize}

\section{Preguntas de investigación}
\noindent
El objetivo de esta investigación es dar respuesta al conjunto de preguntas formuladas aplicando diferentes técnicas de visualización y transformación de datos.

%%%%%%%%%%%%%%%%%%%%%%%%%%%%%%%%%%%%%%%%%
%   Pregunta 1
%%%%%%%%%%%%%%%%%%%%%%%%%%%%%%%%%%%%%%%%%
\subsection{Representa la gráfica \texttt{price} vs. \texttt{caratage} y la gráfica \texttt{log(price)} vs \texttt{caratage}. Decide que variable de respuesta es mejor utilizar.}
\label{subsec:question-1}

En primer lugar, representaremos las dos gráficas

%%%%%%%%%%%%%%%%%%%%%%%%%%%%%%%%%%%%%%%%%
%   Pregunta 2
%%%%%%%%%%%%%%%%%%%%%%%%%%%%%%%%%%%%%%%%%
\subsection{Encuentra una manera adecuada de incluir, además de \texttt{caratage}, la otra información categórica disponible: \texttt{clarity}, \texttt{color} y \texttt{certificate}. Comenta el modelo que se ha ajustado, realizando un análisis básico de los residuos.}
\label{subsec:question-2}


%%%%%%%%%%%%%%%%%%%%%%%%%%%%%%%%%%%%%%%%%
%   Pregunta 3
%%%%%%%%%%%%%%%%%%%%%%%%%%%%%%%%%%%%%%%%%
\subsection{Crea dos acciones para remediar el resultado anterior}
\label{subsec:question-3}

\subsubsection{Crea una nueva variable categórica para segregar las piedras de acuerdo al valor de \texttt{caratage}.}

\subsubsection{Añade el cuadrado de \texttt{caratage} como una nueva variable explicatoria.}



%%%%%%%%%%%%%%%%%%%%%%%%%%%%%%%%%%%%%%%%%
%   Pregunta 4
%%%%%%%%%%%%%%%%%%%%%%%%%%%%%%%%%%%%%%%%%
\subsection{¿Cuál de las acciones anteriores prefieres y por qué? Justifica la respuesta en términos de interpretabilidad y validez de los supuestos.}
\label{subsec:question-4}



%%%%%%%%%%%%%%%%%%%%%%%%%%%%%%%%%%%%%%%%%
\begin{thebibliography}{9}
\bibitem{bortner1969short}
    Journal of chronic diseases (Bortner, Rayman W)
    \newblock {\em A short rating scale as a potential measure of pattern A behavior}, 1969.

\end{thebibliography}

\end{document}



